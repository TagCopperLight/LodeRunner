\chapter{Évaluation Expérimentale}
\label{cp:evaluation}

Afin de valider les performances de notre stratégie, nous avons utilisé un script \texttt{sh}, qui permet de lancer plusieurs parties et de récupérer les résultats.
Ce script nous renvoie, pour un niveau donnée et un nombre de parties donné, le pourcentage de victoire, et le nombre moyen de déplacements effectués par le runner.

\section{Niveau 0}

\subsection{Résultats}

\begin{table}[!htpb]
    \begin{tabularx}{\textwidth}{lXrr}
        \toprule
        \multicolumn{2}{c}{\textbf{Victoires}} & \multicolumn{2}{c}{\textbf{Défaites}} \\
        \cmidrule(lr){1-2} \cmidrule(lr){3-4}
        Pourcentage & Moyenne de déplacements & Pourcentage & Moyenne de déplacements \\
        \midrule
        100\% & 0 & 0\% & 0 \\
        \bottomrule
    \end{tabularx}
    \caption{Résultats pour le niveau 0}
    \label{tab:res-niveau-0}
\end{table}

\subsection{Analyse des défaites}

Il n'y a pas de défaites pour le niveau 0.

\section{Niveau 1}

\subsection{Résultats}

\begin{table}[!htpb]
    \begin{tabularx}{\textwidth}{lXrr}
        \toprule
        \multicolumn{2}{c}{\textbf{Victoires}} & \multicolumn{2}{c}{\textbf{Défaites}} \\
        \cmidrule(lr){1-2} \cmidrule(lr){3-4}
        Pourcentage & Moyenne de déplacements & Pourcentage & Moyenne de déplacements \\
        \midrule
        100\% & 0 & 0\% & 0 \\
        \bottomrule
    \end{tabularx}
    \caption{Résultats pour le niveau 1}
    \label{tab:res-niveau-1}
\end{table}

\subsection{Analyse des défaites}

Il n'y a pas de défaites pour le niveau 1.

\section{Niveau 2}

\subsection{Résultats}

\begin{table}[!htpb]
    \begin{tabularx}{\textwidth}{lXrr}
        \toprule
        \multicolumn{2}{c}{\textbf{Victoires}} & \multicolumn{2}{c}{\textbf{Défaites}} \\
        \cmidrule(lr){1-2} \cmidrule(lr){3-4}
        Pourcentage & Moyenne de déplacements & Pourcentage & Moyenne de déplacements \\
        \midrule
        100\% & 0 & 0\% & 0 \\
        \bottomrule
    \end{tabularx}
    \caption{Résultats pour le niveau 2}
    \label{tab:res-niveau-2}
\end{table}

\subsection{Analyse des défaites}

Il n'y a pas de défaites pour le niveau 2.

\section{Niveau 3}

\subsection{Résultats}

\begin{table}[!htpb]
    \begin{tabularx}{\textwidth}{lXrr}
        \toprule
        \multicolumn{2}{c}{\textbf{Victoires}} & \multicolumn{2}{c}{\textbf{Défaites}} \\
        \cmidrule(lr){1-2} \cmidrule(lr){3-4}
        Pourcentage & Moyenne de déplacements & Pourcentage & Moyenne de déplacements \\
        \midrule
        100\% & 0 & 0\% & 0 \\
        \bottomrule
    \end{tabularx}
    \caption{Résultats pour le niveau 3}
    \label{tab:res-niveau-3}
\end{table}

\subsection{Analyse des défaites}

Il n'y a pas de défaites pour le niveau 3.

\section{Niveau 4}

\subsection{Résultats}

\begin{table}[!htpb]
    \begin{tabularx}{\textwidth}{lXrr}
        \toprule
        \multicolumn{2}{c}{\textbf{Victoires}} & \multicolumn{2}{c}{\textbf{Défaites}} \\
        \cmidrule(lr){1-2} \cmidrule(lr){3-4}
        Pourcentage & Moyenne de déplacements & Pourcentage & Moyenne de déplacements \\
        \midrule
        100\% & 0 & 0\% & 0 \\
        \bottomrule
    \end{tabularx}
    \caption{Résultats pour le niveau 4}
    \label{tab:res-niveau-4}
\end{table}

\subsection{Analyse des défaites}

Il n'y a pas de défaites pour le niveau 4.

\section{Niveau supplémentaire}

\subsection{Résultats}

\begin{table}[!htpb]
    \begin{tabularx}{\textwidth}{lXrr}
        \toprule
        \multicolumn{2}{c}{\textbf{Victoires}} & \multicolumn{2}{c}{\textbf{Défaites}} \\
        \cmidrule(lr){1-2} \cmidrule(lr){3-4}
        Pourcentage & Moyenne de déplacements & Pourcentage & Moyenne de déplacements \\
        \midrule
        100\% & 0 & 0\% & 0 \\
        \bottomrule
    \end{tabularx}
    \caption{Résultats pour le niveau supplémentaire}
    \label{tab:res-niveau-supplementaire}
\end{table}

\subsection{Analyse des défaites}

Il n'y a pas de défaites pour le niveau supplémentaire.

\section{Conclusion}

Nous avons pu constater que notre stratégie est efficace pour les niveaux 0 à 4, ainsi que pour le niveau supplémentaire.