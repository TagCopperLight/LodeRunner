\chapter{Conclusion}
\label{cp:conclusion}

Ce projet autour du jeu Lode Runner nous a permis de concevoir et d'implémenter une intelligence artificielle pour jouer à ce jeu.
Malgré des contraintes, nous avons pu élaborer une stratégie robuste basée sur l'algorithme A* pour gérer au mieux les situations rencontrées.
\newline
L'évaluation expérimentale a démontré l'efficacité de notre IA sur des niveaux variés, atteignant un taux de réussite de 100\% pour les niveaux les plus simples et maintenant des performances honorables face à des niveaux plus difficiles.
Cela reflète d'une cohérence entre la conception algorithmique de notre stratégie et son application.
\newline
Ces résultats montrent aussi les limites de notre IA, notamment face à des niveaux plus complexes, où des améliorations pourraient être apportées pour gérer des situations avec davanatge d'ennemis.
\newline\newline
Même si j'étais en filière MPI l'année dernière, j'ai apprécié travailler sur ce projet, qui m'a permis de pousser mes compétences en algorithmique et en programmation.
\newline
Enfin, la rédaction de ce rapport m'a beaucoup appris, car c'est la première fois que je rédige un rapport aussi long et complet.
J'ai aussi pu m'initier à \LaTeX, que je n'avais jamais utilisé auparavant.