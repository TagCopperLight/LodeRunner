\chapter{Introduction}
\label{cp:introduction}
Dans le cadre du module Algorithme 1, nous avons pu appliquer les concepts étudiés en cours en réalisant un projet. Ce projet consiste à réaliser une IA capable de jouer au jeu Lode Runner. C'est un jeu d'arcade qui se déroule sur une carte en deux dimensions où le joueur doit récupérer des bonus tout en évitant des ennemis. 
\newline\newline
L'objectif du projet est de développer une IA capable de pouvoir terminer n'importe quel niveau du jeu. Cependant, des limitations ont été imposées : l'IA ne connaît pas les actions futures des ennemis et ne peut pas mémoriser des informations d'un tour à l'autre. Ce cadre impose une stratégie approfondie sur la position actuelle afin de ne pas faire un mouvement qui serait mortel plus tard.
\newline\newline
Au cours du développement, j'ai rencontré plusieurs défis. L'absence de mémorisation empêche d'utiliser une stratégie qui prend trop de temps de calcul ou de pouvoir créer une stratégie sur plusieurs tours, tandis que le fait de ne pas connaître les prochains coups des ennemis empêche d'explorer un “arbre des positions”. Par ailleurs, l'implémentation en C, avec ses contraintes de gestion manuelle de la mémoire, a ajouté une dimension technique non négligeable.
\newline\newline
Malgré ces contraintes, j'ai pu élaborer une IA fonctionnelle et performante (au moins sur les niveaux disponibles).
\newline\newline
Ce compte rendu présente ma démarche de développement, la stratégie utilisée et les résultats obtenus.